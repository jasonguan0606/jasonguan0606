% Options for packages loaded elsewhere
\PassOptionsToPackage{unicode}{hyperref}
\PassOptionsToPackage{hyphens}{url}
%
\documentclass[
]{article}
\usepackage{amsmath,amssymb}
\usepackage{iftex}
\ifPDFTeX
  \usepackage[T1]{fontenc}
  \usepackage[utf8]{inputenc}
  \usepackage{textcomp} % provide euro and other symbols
\else % if luatex or xetex
  \usepackage{unicode-math} % this also loads fontspec
  \defaultfontfeatures{Scale=MatchLowercase}
  \defaultfontfeatures[\rmfamily]{Ligatures=TeX,Scale=1}
\fi
\usepackage{lmodern}
\ifPDFTeX\else
  % xetex/luatex font selection
\fi
% Use upquote if available, for straight quotes in verbatim environments
\IfFileExists{upquote.sty}{\usepackage{upquote}}{}
\IfFileExists{microtype.sty}{% use microtype if available
  \usepackage[]{microtype}
  \UseMicrotypeSet[protrusion]{basicmath} % disable protrusion for tt fonts
}{}
\makeatletter
\@ifundefined{KOMAClassName}{% if non-KOMA class
  \IfFileExists{parskip.sty}{%
    \usepackage{parskip}
  }{% else
    \setlength{\parindent}{0pt}
    \setlength{\parskip}{6pt plus 2pt minus 1pt}}
}{% if KOMA class
  \KOMAoptions{parskip=half}}
\makeatother
\usepackage{xcolor}
\usepackage[margin=1in]{geometry}
\usepackage{color}
\usepackage{fancyvrb}
\newcommand{\VerbBar}{|}
\newcommand{\VERB}{\Verb[commandchars=\\\{\}]}
\DefineVerbatimEnvironment{Highlighting}{Verbatim}{commandchars=\\\{\}}
% Add ',fontsize=\small' for more characters per line
\usepackage{framed}
\definecolor{shadecolor}{RGB}{248,248,248}
\newenvironment{Shaded}{\begin{snugshade}}{\end{snugshade}}
\newcommand{\AlertTok}[1]{\textcolor[rgb]{0.94,0.16,0.16}{#1}}
\newcommand{\AnnotationTok}[1]{\textcolor[rgb]{0.56,0.35,0.01}{\textbf{\textit{#1}}}}
\newcommand{\AttributeTok}[1]{\textcolor[rgb]{0.13,0.29,0.53}{#1}}
\newcommand{\BaseNTok}[1]{\textcolor[rgb]{0.00,0.00,0.81}{#1}}
\newcommand{\BuiltInTok}[1]{#1}
\newcommand{\CharTok}[1]{\textcolor[rgb]{0.31,0.60,0.02}{#1}}
\newcommand{\CommentTok}[1]{\textcolor[rgb]{0.56,0.35,0.01}{\textit{#1}}}
\newcommand{\CommentVarTok}[1]{\textcolor[rgb]{0.56,0.35,0.01}{\textbf{\textit{#1}}}}
\newcommand{\ConstantTok}[1]{\textcolor[rgb]{0.56,0.35,0.01}{#1}}
\newcommand{\ControlFlowTok}[1]{\textcolor[rgb]{0.13,0.29,0.53}{\textbf{#1}}}
\newcommand{\DataTypeTok}[1]{\textcolor[rgb]{0.13,0.29,0.53}{#1}}
\newcommand{\DecValTok}[1]{\textcolor[rgb]{0.00,0.00,0.81}{#1}}
\newcommand{\DocumentationTok}[1]{\textcolor[rgb]{0.56,0.35,0.01}{\textbf{\textit{#1}}}}
\newcommand{\ErrorTok}[1]{\textcolor[rgb]{0.64,0.00,0.00}{\textbf{#1}}}
\newcommand{\ExtensionTok}[1]{#1}
\newcommand{\FloatTok}[1]{\textcolor[rgb]{0.00,0.00,0.81}{#1}}
\newcommand{\FunctionTok}[1]{\textcolor[rgb]{0.13,0.29,0.53}{\textbf{#1}}}
\newcommand{\ImportTok}[1]{#1}
\newcommand{\InformationTok}[1]{\textcolor[rgb]{0.56,0.35,0.01}{\textbf{\textit{#1}}}}
\newcommand{\KeywordTok}[1]{\textcolor[rgb]{0.13,0.29,0.53}{\textbf{#1}}}
\newcommand{\NormalTok}[1]{#1}
\newcommand{\OperatorTok}[1]{\textcolor[rgb]{0.81,0.36,0.00}{\textbf{#1}}}
\newcommand{\OtherTok}[1]{\textcolor[rgb]{0.56,0.35,0.01}{#1}}
\newcommand{\PreprocessorTok}[1]{\textcolor[rgb]{0.56,0.35,0.01}{\textit{#1}}}
\newcommand{\RegionMarkerTok}[1]{#1}
\newcommand{\SpecialCharTok}[1]{\textcolor[rgb]{0.81,0.36,0.00}{\textbf{#1}}}
\newcommand{\SpecialStringTok}[1]{\textcolor[rgb]{0.31,0.60,0.02}{#1}}
\newcommand{\StringTok}[1]{\textcolor[rgb]{0.31,0.60,0.02}{#1}}
\newcommand{\VariableTok}[1]{\textcolor[rgb]{0.00,0.00,0.00}{#1}}
\newcommand{\VerbatimStringTok}[1]{\textcolor[rgb]{0.31,0.60,0.02}{#1}}
\newcommand{\WarningTok}[1]{\textcolor[rgb]{0.56,0.35,0.01}{\textbf{\textit{#1}}}}
\usepackage{graphicx}
\makeatletter
\def\maxwidth{\ifdim\Gin@nat@width>\linewidth\linewidth\else\Gin@nat@width\fi}
\def\maxheight{\ifdim\Gin@nat@height>\textheight\textheight\else\Gin@nat@height\fi}
\makeatother
% Scale images if necessary, so that they will not overflow the page
% margins by default, and it is still possible to overwrite the defaults
% using explicit options in \includegraphics[width, height, ...]{}
\setkeys{Gin}{width=\maxwidth,height=\maxheight,keepaspectratio}
% Set default figure placement to htbp
\makeatletter
\def\fps@figure{htbp}
\makeatother
\setlength{\emergencystretch}{3em} % prevent overfull lines
\providecommand{\tightlist}{%
  \setlength{\itemsep}{0pt}\setlength{\parskip}{0pt}}
\setcounter{secnumdepth}{-\maxdimen} % remove section numbering
\ifLuaTeX
  \usepackage{selnolig}  % disable illegal ligatures
\fi
\usepackage{bookmark}
\IfFileExists{xurl.sty}{\usepackage{xurl}}{} % add URL line breaks if available
\urlstyle{same}
\hypersetup{
  pdftitle={Untitled},
  pdfauthor={Jason},
  hidelinks,
  pdfcreator={LaTeX via pandoc}}

\title{Untitled}
\author{Jason}
\date{2024-12-17}

\begin{document}
\maketitle

\subsection{1.
从流行病到全球大流行}\label{ux4eceux6d41ux884cux75c5ux5230ux5168ux7403ux5927ux6d41ux884c}

2019年12月,中国武汉地区首次发现COVID-19冠状病毒。到2020年3月11日,世界卫生组织(WHO)将COVID-19疫情列为大流行病。在这几个月中,伊朗、韩国和意大利都爆发了重大疫情。

我们知道,COVID-19
通过呼吸道飞沫传播,如通过咳嗽、打喷嚏或说话。但是,病毒在全球传播的速度有多快?我们是否能看到国家范围内的政策(如关闭和隔离)所产生的任何影响?

幸运的是,世界各地的组织一直在收集数据,以便各国政府能够监测和了解这一流行病。值得注意的是,约翰霍普金斯大学系统科学与工程中心创建了一个数据整合系统,从世界卫生组织、美国疾病控制和预防中心(CDC)以及多个国家的卫生部等来源收集数据。

在本文中,将对 COVID-19
病毒爆发最初几周的数据进行可视化处理,以了解该病毒是在什么时候成为全球大流行病的。

请注意,有关 COVID-19 的信息和数据经常更新。本项目中使用的数据是 2020 年
3 月 17 日提取的,不应被视为现有的最新数据。

\begin{Shaded}
\begin{Highlighting}[]
\CommentTok{\# 加载readr、ggplot2和dplyr包}
\FunctionTok{library}\NormalTok{(readr)}
\FunctionTok{library}\NormalTok{(ggplot2)}
\FunctionTok{library}\NormalTok{(dplyr)}
\end{Highlighting}
\end{Shaded}

\begin{verbatim}
## Warning: 程序包'dplyr'是用R版本4.4.2 来建造的
\end{verbatim}

\begin{verbatim}
## 
## 载入程序包:'dplyr'
\end{verbatim}

\begin{verbatim}
## The following objects are masked from 'package:stats':
## 
##     filter, lag
\end{verbatim}

\begin{verbatim}
## The following objects are masked from 'package:base':
## 
##     intersect, setdiff, setequal, union
\end{verbatim}

\begin{Shaded}
\begin{Highlighting}[]
\CommentTok{\# 将数据集"datasets/confirmed\_cases\_worldwide.csv"读入到confirmed\_cases\_worldwide变量中}
\NormalTok{confirmed\_cases\_worldwide }\OtherTok{\textless{}{-}} \FunctionTok{read\_csv}\NormalTok{(}\StringTok{"C:/Users/25560/Desktop/Visualizing COVID{-}19/datasets/confirmed\_cases\_worldwide.csv"}\NormalTok{)}
\end{Highlighting}
\end{Shaded}

\begin{verbatim}
## Rows: 56 Columns: 2
\end{verbatim}

\begin{verbatim}
## -- Column specification --------------------------------------------------------
## Delimiter: ","
## dbl  (1): cum_cases
## date (1): date
## 
## i Use `spec()` to retrieve the full column specification for this data.
## i Specify the column types or set `show_col_types = FALSE` to quiet this message.
\end{verbatim}

\begin{Shaded}
\begin{Highlighting}[]
\CommentTok{\# 打印出confirmed\_cases\_worldwide的内容}
\NormalTok{confirmed\_cases\_worldwide}
\end{Highlighting}
\end{Shaded}

\begin{verbatim}
## # A tibble: 56 x 2
##    date       cum_cases
##    <date>         <dbl>
##  1 2020-01-22       555
##  2 2020-01-23       653
##  3 2020-01-24       941
##  4 2020-01-25      1434
##  5 2020-01-26      2118
##  6 2020-01-27      2927
##  7 2020-01-28      5578
##  8 2020-01-29      6166
##  9 2020-01-30      8234
## 10 2020-01-31      9927
## # i 46 more rows
\end{verbatim}

\subsection{2.
世界各地的确诊病例}\label{ux4e16ux754cux5404ux5730ux7684ux786eux8bcaux75c5ux4f8b}

上表显示了按日期分列的全球 COVID-19
累计确诊病例。光看表格中的数字很难了解疫情的规模和增长情况。让我们绘制一张线图来直观地显示全球确诊病例。

\begin{Shaded}
\begin{Highlighting}[]
\CommentTok{\# 绘制累计病例与日期的折线图}
\CommentTok{\# 标记y轴}
\FunctionTok{ggplot}\NormalTok{(confirmed\_cases\_worldwide, }\FunctionTok{aes}\NormalTok{(}\AttributeTok{x =}\NormalTok{ date, }\AttributeTok{y =}\NormalTok{ cum\_cases)) }\SpecialCharTok{+}
  \FunctionTok{geom\_line}\NormalTok{() }\SpecialCharTok{+}                       \CommentTok{\# 这行添加了折线图层到图表中。}
  \FunctionTok{ylab}\NormalTok{(}\StringTok{"累计确诊病例数"}\NormalTok{)              }\CommentTok{\# 这行设置了y轴的标签为指定的文本。}
\end{Highlighting}
\end{Shaded}

\includegraphics{探索性数据分析-管培哲-24349134_files/figure-latex/unnamed-chunk-2-1.pdf}

\subsection{3.
中国与世界其他国家的比较}\label{ux4e2dux56fdux4e0eux4e16ux754cux5176ux4ed6ux56fdux5bb6ux7684ux6bd4ux8f83}

图中的 y 轴非常吓人,全球确诊病例总数已接近 20
万例。除此之外,还出现了一些奇怪的现象:二月中旬出现了一个奇怪的跳跃,然后新增病例的速度放缓了一段时间,三月份又加快了。我们需要深入调查,看看到底发生了什么。

疫情爆发初期,COVID-19
病例主要集中在中国。让我们分别绘制中国和世界其他地区的 COVID-19
确诊病例,看看是否能给我们一些启示。

在未来的任务中,我们将以这幅图为基础。在接下来的任务中,有一点很重要,那就是要在
ggplot 的线条几何图形中添加美观效果,而不是让它们成为全局美观效果。

\begin{Shaded}
\begin{Highlighting}[]
\CommentTok{\# 读取数据集}
\NormalTok{confirmed\_cases\_china\_vs\_world }\OtherTok{\textless{}{-}} \FunctionTok{read\_csv}\NormalTok{(}\StringTok{"C:/Users/25560/Desktop/Visualizing COVID{-}19/datasets/confirmed\_cases\_china\_vs\_world.csv"}\NormalTok{)}
\end{Highlighting}
\end{Shaded}

\begin{verbatim}
## Rows: 112 Columns: 4
## -- Column specification --------------------------------------------------------
## Delimiter: ","
## chr  (1): is_china
## dbl  (2): cases, cum_cases
## date (1): date
## 
## i Use `spec()` to retrieve the full column specification for this data.
## i Specify the column types or set `show_col_types = FALSE` to quiet this message.
\end{verbatim}

\begin{Shaded}
\begin{Highlighting}[]
\CommentTok{\# 查看数据集的结构和前几行内容}
\FunctionTok{glimpse}\NormalTok{(confirmed\_cases\_china\_vs\_world)}
\end{Highlighting}
\end{Shaded}

\begin{verbatim}
## Rows: 112
## Columns: 4
## $ is_china  <chr> "China", "China", "China", "China", "China", "China", "China~
## $ date      <date> 2020-01-22, 2020-01-23, 2020-01-24, 2020-01-25, 2020-01-26,~
## $ cases     <dbl> 548, 95, 277, 486, 669, 802, 2632, 578, 2054, 1661, 2089, 47~
## $ cum_cases <dbl> 548, 643, 920, 1406, 2075, 2877, 5509, 6087, 8141, 9802, 118~
\end{verbatim}

\begin{Shaded}
\begin{Highlighting}[]
\CommentTok{\# 绘制累计病例与日期的折线图,并根据是否为中国来着色}
\CommentTok{\# 在线条geom中定义美学属性}
\NormalTok{plt\_cum\_confirmed\_cases\_china\_vs\_world }\OtherTok{\textless{}{-}} \FunctionTok{ggplot}\NormalTok{(confirmed\_cases\_china\_vs\_world) }\SpecialCharTok{+}
  \FunctionTok{geom\_line}\NormalTok{(}\FunctionTok{aes}\NormalTok{(}\AttributeTok{x =}\NormalTok{ date, }\AttributeTok{y =}\NormalTok{ cum\_cases, }\AttributeTok{color =}\NormalTok{ is\_china)) }\SpecialCharTok{+}  \CommentTok{\# 使用日期作为x轴,累计病例数作为y轴,并根据is\_china字段的颜色区分中国与其他地区}
  \FunctionTok{ylab}\NormalTok{(}\StringTok{"累计确诊病例数"}\NormalTok{)  }\CommentTok{\# 设置y轴标签}

\CommentTok{\# 显示图表}
\NormalTok{plt\_cum\_confirmed\_cases\_china\_vs\_world}
\end{Highlighting}
\end{Shaded}

\includegraphics{探索性数据分析-管培哲-24349134_files/figure-latex/unnamed-chunk-3-1.pdf}

\subsection{4. 注释一下}\label{ux6ce8ux91caux4e00ux4e0b}

两条线的形状截然不同。2 月份,大多数病例发生在中国。到了 3
月份,情况发生了变化,疫情真正成为全球性爆发:3 月 14
日前后,中国境外的病例总数超过了中国境内的病例总数。这是在世界卫生组织宣布疫情大流行几天之后。

疫情爆发期间还发生了其他一些标志性事件。例如,2020 年 2 月 13
日中国线的大幅跳水并不仅仅是疫情的糟糕一天;中国改变了当天报告数据的方式(接受
CT 扫描作为 COVID-19 的证据,而不仅仅是实验室检测)。

通过注释这样的事件,我们可以更好地解读情节的变化。

\begin{Shaded}
\begin{Highlighting}[]
\CommentTok{\# 创建一个数据框who\_events,记录了重要的WHO事件及其日期}
\NormalTok{who\_events }\OtherTok{\textless{}{-}} \FunctionTok{tribble}\NormalTok{(}
  \SpecialCharTok{\textasciitilde{}}\NormalTok{ date,                    }\SpecialCharTok{\textasciitilde{}}\NormalTok{ event,}
  \StringTok{"2020{-}01{-}30"}\NormalTok{,             }\StringTok{"全球卫生紧急状态宣布"}\NormalTok{,}
  \StringTok{"2020{-}03{-}11"}\NormalTok{,             }\StringTok{"大流行宣布"}\NormalTok{,}
  \StringTok{"2020{-}02{-}13"}\NormalTok{,             }\StringTok{"中国报告方式变更"}
\NormalTok{) }\SpecialCharTok{\%\textgreater{}\%}
  \FunctionTok{mutate}\NormalTok{(}\AttributeTok{date =} \FunctionTok{as.Date}\NormalTok{(date))  }\CommentTok{\# 将日期列转换为Date类型}

\CommentTok{\# 在之前创建的累计确诊病例数折线图基础上,}
\CommentTok{\# 添加垂直虚线,x轴截距为who\_events中的日期,}
\CommentTok{\# 并在y轴100000位置添加文本标签,内容为who\_events中的事件描述}
\NormalTok{plt\_cum\_confirmed\_cases\_china\_vs\_world }\SpecialCharTok{+}
  \FunctionTok{geom\_vline}\NormalTok{(}\FunctionTok{aes}\NormalTok{(}\AttributeTok{xintercept =}\NormalTok{ date), }\AttributeTok{data =}\NormalTok{ who\_events, }\AttributeTok{linetype =} \StringTok{"dashed"}\NormalTok{) }\SpecialCharTok{+}  \CommentTok{\# 添加垂直虚线标记关键事件日期}
  \FunctionTok{geom\_text}\NormalTok{(}\FunctionTok{aes}\NormalTok{(}\AttributeTok{x =}\NormalTok{ date, }\AttributeTok{y =} \FloatTok{1e5}\NormalTok{, }\AttributeTok{label =}\NormalTok{ event), }\AttributeTok{data =}\NormalTok{ who\_events, }\AttributeTok{vjust =} \SpecialCharTok{{-}}\FloatTok{0.5}\NormalTok{)  }\CommentTok{\# 在每个事件日期上添加文本标签,vjust调整文本的垂直对齐}
\end{Highlighting}
\end{Shaded}

\includegraphics{探索性数据分析-管培哲-24349134_files/figure-latex/unnamed-chunk-4-1.pdf}

\subsection{5. 添加趋势线}\label{ux6dfbux52a0ux8d8bux52bfux7ebf}

在评估未来问题的严重程度时,我们需要衡量案例数量的增长速度。一个很好的起点是看案例的增长速度是快于还是慢于线性增长速度。

2020 年 2 月 13
日前后,随着中国报告的变化,病例明显激增。然而,几天后,中国的病例增长放缓。如何描述
2020 年 2 月 15 日之后 COVID-19 在中国的增长情况?

\begin{Shaded}
\begin{Highlighting}[]
\CommentTok{\# 筛选出中国从2月15日以后的数据}
\NormalTok{china\_after\_feb15 }\OtherTok{\textless{}{-}}\NormalTok{ confirmed\_cases\_china\_vs\_world }\SpecialCharTok{\%\textgreater{}\%}
  \FunctionTok{filter}\NormalTok{(is\_china }\SpecialCharTok{==} \StringTok{"China"}\NormalTok{, date }\SpecialCharTok{\textgreater{}=} \FunctionTok{as.Date}\NormalTok{(}\StringTok{"2020{-}02{-}15"}\NormalTok{))}

\CommentTok{\# 使用china\_after\_feb15数据集,绘制累计病例与日期的折线图}
\CommentTok{\# 添加一条使用线性回归方法计算的平滑趋势线,不显示误差范围}
\FunctionTok{ggplot}\NormalTok{(china\_after\_feb15, }\FunctionTok{aes}\NormalTok{(}\AttributeTok{x =}\NormalTok{ date, }\AttributeTok{y =}\NormalTok{ cum\_cases)) }\SpecialCharTok{+}
  \FunctionTok{geom\_line}\NormalTok{() }\SpecialCharTok{+}  \CommentTok{\# 绘制折线图}
  \FunctionTok{geom\_smooth}\NormalTok{(}\AttributeTok{method =} \StringTok{"lm"}\NormalTok{, }\AttributeTok{se =} \ConstantTok{FALSE}\NormalTok{) }\SpecialCharTok{+}  \CommentTok{\# 添加线性回归平滑趋势线,不显示置信区间(误差条)}
  \FunctionTok{ylab}\NormalTok{(}\StringTok{"累计确诊病例数"}\NormalTok{)  }\CommentTok{\# 设置y轴标签为“累计确诊病例数”}
\end{Highlighting}
\end{Shaded}

\begin{verbatim}
## `geom_smooth()` using formula = 'y ~ x'
\end{verbatim}

\includegraphics{探索性数据分析-管培哲-24349134_files/figure-latex/unnamed-chunk-5-1.pdf}

\subsection{6.
世界其他地方呢?}\label{ux4e16ux754cux5176ux4ed6ux5730ux65b9ux5462}

从上图可以看出,中国的增长率低于线性增长率。这是个好消息,因为它表明中国至少在一定程度上遏制了
2 月底和 3 月初的病毒传播。

与线性增长相比,世界其他国家的情况如何?

\begin{Shaded}
\begin{Highlighting}[]
\CommentTok{\# 筛选出数据集中不属于中国的记录}
\NormalTok{not\_china }\OtherTok{\textless{}{-}}\NormalTok{ confirmed\_cases\_china\_vs\_world }\SpecialCharTok{\%\textgreater{}\%}
  \FunctionTok{filter}\NormalTok{(is\_china }\SpecialCharTok{==} \StringTok{"Not China"}\NormalTok{)}

\CommentTok{\# 使用not\_china数据集,绘制累计病例与日期的折线图}
\CommentTok{\# 添加一条使用线性回归方法计算的平滑趋势线,不显示误差范围}
\NormalTok{plt\_not\_china\_trend\_lin }\OtherTok{\textless{}{-}} \FunctionTok{ggplot}\NormalTok{(not\_china, }\FunctionTok{aes}\NormalTok{(}\AttributeTok{x =}\NormalTok{ date, }\AttributeTok{y =}\NormalTok{ cum\_cases)) }\SpecialCharTok{+}
  \FunctionTok{geom\_line}\NormalTok{() }\SpecialCharTok{+}  \CommentTok{\# 绘制折线图层}
  \FunctionTok{geom\_smooth}\NormalTok{(}\AttributeTok{method =} \StringTok{"lm"}\NormalTok{, }\AttributeTok{se =} \ConstantTok{FALSE}\NormalTok{) }\SpecialCharTok{+}  \CommentTok{\# 添加线性回归平滑趋势线,不显示置信区间(误差条)}
  \FunctionTok{ylab}\NormalTok{(}\StringTok{"累计确诊病例数"}\NormalTok{)  }\CommentTok{\# 设置y轴标签为“累计确诊病例数”}

\CommentTok{\# 显示绘图结果}
\NormalTok{plt\_not\_china\_trend\_lin }
\end{Highlighting}
\end{Shaded}

\begin{verbatim}
## `geom_smooth()` using formula = 'y ~ x'
\end{verbatim}

\includegraphics{探索性数据分析-管培哲-24349134_files/figure-latex/unnamed-chunk-6-1.pdf}

\subsection{7. 添加对数刻度}\label{ux6dfbux52a0ux5bf9ux6570ux523bux5ea6}

从上图中我们可以看出,直线的拟合效果并不好,世界其他地区的增长速度要比直线快得多。如果我们在
y 轴上添加一个对数刻度呢?

\begin{Shaded}
\begin{Highlighting}[]
\CommentTok{\# 在之前创建的图表基础上,修改y轴为对数尺度}
\NormalTok{plt\_not\_china\_trend\_lin }\SpecialCharTok{+} 
  \FunctionTok{scale\_y\_log10}\NormalTok{()  }\CommentTok{\# 将y轴设置为以10为底的对数尺度,以便更清晰地展示累计病例数的增长趋势}
\end{Highlighting}
\end{Shaded}

\begin{verbatim}
## `geom_smooth()` using formula = 'y ~ x'
\end{verbatim}

\includegraphics{探索性数据分析-管培哲-24349134_files/figure-latex/unnamed-chunk-7-1.pdf}

\subsection{8.
中国以外哪些国家受到的冲击最大?}\label{ux4e2dux56fdux4ee5ux5916ux54eaux4e9bux56fdux5bb6ux53d7ux5230ux7684ux51b2ux51fbux6700ux5927}

使用对数标度,我们可以得到更接近数据的拟合结果。从数据科学的角度来看,拟合良好是个好消息。不幸的是,从公共卫生的角度来看,这意味着世界其他地区的
COVID-19 病例正在以指数速度增长,这是一个可怕的消息。

并非所有国家都同样受到 COVID-19
的影响,因此了解世界上哪些国家的问题最严重会有所帮助。在数据集中找到确诊病例数最高的国家。

\begin{Shaded}
\begin{Highlighting}[]
\CommentTok{\# 运行此代码以获取每个国家的确诊病例数据}
\NormalTok{confirmed\_cases\_by\_country }\OtherTok{\textless{}{-}} \FunctionTok{read\_csv}\NormalTok{(}\StringTok{"C:/Users/25560/Desktop/Visualizing COVID{-}19/datasets/confirmed\_cases\_by\_country.csv"}\NormalTok{)}
\end{Highlighting}
\end{Shaded}

\begin{verbatim}
## Rows: 13272 Columns: 5
## -- Column specification --------------------------------------------------------
## Delimiter: ","
## chr  (2): country, province
## dbl  (2): cases, cum_cases
## date (1): date
## 
## i Use `spec()` to retrieve the full column specification for this data.
## i Specify the column types or set `show_col_types = FALSE` to quiet this message.
\end{verbatim}

\begin{Shaded}
\begin{Highlighting}[]
\FunctionTok{glimpse}\NormalTok{(confirmed\_cases\_by\_country)  }\CommentTok{\# 查看数据集的结构和前几行内容}
\end{Highlighting}
\end{Shaded}

\begin{verbatim}
## Rows: 13,272
## Columns: 5
## $ country   <chr> "Afghanistan", "Albania", "Algeria", "Andorra", "Antigua and~
## $ province  <chr> NA, NA, NA, NA, NA, NA, NA, NA, NA, NA, NA, NA, NA, NA, NA, ~
## $ date      <date> 2020-01-22, 2020-01-22, 2020-01-22, 2020-01-22, 2020-01-22,~
## $ cases     <dbl> 0, 0, 0, 0, 0, 0, 0, 0, 0, 0, 0, 0, 0, 0, 0, 0, 0, 0, 0, 0, ~
## $ cum_cases <dbl> 0, 0, 0, 0, 0, 0, 0, 0, 0, 0, 0, 0, 0, 0, 0, 0, 0, 0, 0, 0, ~
\end{verbatim}

\begin{Shaded}
\begin{Highlighting}[]
\CommentTok{\# 按国家分组,汇总计算总病例数,找出累计病例数最多的前7个国家}
\NormalTok{top\_countries\_by\_total\_cases }\OtherTok{\textless{}{-}}\NormalTok{ confirmed\_cases\_by\_country }\SpecialCharTok{\%\textgreater{}\%}
  \FunctionTok{group\_by}\NormalTok{(country) }\SpecialCharTok{\%\textgreater{}\%}  \CommentTok{\# 按国家分组}
  \FunctionTok{summarize}\NormalTok{(}\AttributeTok{total\_cases =} \FunctionTok{max}\NormalTok{(cum\_cases)) }\SpecialCharTok{\%\textgreater{}\%}  \CommentTok{\# 计算每个国家的最大累计病例数作为总病例数}
  \FunctionTok{top\_n}\NormalTok{(}\DecValTok{7}\NormalTok{, total\_cases)  }\CommentTok{\# 找出总病例数最多的前7个国家}

\CommentTok{\# 显示结果}
\NormalTok{top\_countries\_by\_total\_cases}
\end{Highlighting}
\end{Shaded}

\begin{verbatim}
## # A tibble: 7 x 2
##   country      total_cases
##   <chr>              <dbl>
## 1 France              7699
## 2 Germany             9257
## 3 Iran               16169
## 4 Italy              31506
## 5 Korea, South        8320
## 6 Spain              11748
## 7 US                  6421
\end{verbatim}

\subsection{9. 截至 2020 年 3
月中旬受影响最严重的国家分布图}\label{ux622aux81f3-2020-ux5e74-3-ux6708ux4e2dux65ecux53d7ux5f71ux54cdux6700ux4e25ux91cdux7684ux56fdux5bb6ux5206ux5e03ux56fe}

尽管疫情首先在中国发现,但上表中只有一个东亚国家(韩国)。所列国家中有四个(法国、德国、意大利和西班牙)位于欧洲,并且与欧洲接壤。为了获得更多信息,我们可以绘制这些国家的确诊病例随时间变化的曲线。

如果您想继续制作可视化图表或查找截至目前受影响最严重的国家,可以使用现有的最新数据进行分析
here.

\begin{Shaded}
\begin{Highlighting}[]
\CommentTok{\# confirmed\_cases\_top7\_outside\_china.csv读取数据集}
\NormalTok{confirmed\_cases\_top7\_outside\_china }\OtherTok{\textless{}{-}} \FunctionTok{read\_csv}\NormalTok{(}\StringTok{"C:/Users/25560/Desktop/Visualizing COVID{-}19/datasets/confirmed\_cases\_top7\_outside\_china.csv"}\NormalTok{)}
\end{Highlighting}
\end{Shaded}

\begin{verbatim}
## Rows: 2030 Columns: 3
## -- Column specification --------------------------------------------------------
## Delimiter: ","
## chr  (1): country
## dbl  (1): cum_cases
## date (1): date
## 
## i Use `spec()` to retrieve the full column specification for this data.
## i Specify the column types or set `show_col_types = FALSE` to quiet this message.
\end{verbatim}

\begin{Shaded}
\begin{Highlighting}[]
\CommentTok{\# 查看confirmed\_cases\_top7\_outside\_china数据集的内容}
\FunctionTok{glimpse}\NormalTok{(confirmed\_cases\_top7\_outside\_china)}
\end{Highlighting}
\end{Shaded}

\begin{verbatim}
## Rows: 2,030
## Columns: 3
## $ country   <chr> "Germany", "Iran", "Italy", "Korea, South", "Spain", "US", "~
## $ date      <date> 2020-02-18, 2020-02-18, 2020-02-18, 2020-02-18, 2020-02-18,~
## $ cum_cases <dbl> 16, 0, 3, 31, 2, 13, 13, 13, 13, 13, 13, 13, 13, 13, 13, 13,~
\end{verbatim}

\begin{Shaded}
\begin{Highlighting}[]
\CommentTok{\# 使用confirmed\_cases\_top7\_outside\_china数据集,绘制累计病例与日期的折线图,并根据国家着色}
\FunctionTok{ggplot}\NormalTok{(confirmed\_cases\_top7\_outside\_china, }\FunctionTok{aes}\NormalTok{(date, cum\_cases, }\AttributeTok{color =}\NormalTok{ country)) }\SpecialCharTok{+}
  \FunctionTok{geom\_line}\NormalTok{() }\SpecialCharTok{+}
  \FunctionTok{ylab}\NormalTok{(}\StringTok{"累计确诊病例数"}\NormalTok{)}
\end{Highlighting}
\end{Shaded}

\includegraphics{探索性数据分析-管培哲-24349134_files/figure-latex/unnamed-chunk-9-1.pdf}

\end{document}
